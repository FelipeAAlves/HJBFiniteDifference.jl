
\documentclass[english]{article}
\usepackage[T1]{fontenc}
\usepackage[latin9]{inputenc}
\usepackage{color}
\usepackage{amsmath}
\usepackage{amssymb}
\usepackage{esint}
\usepackage{babel}
\begin{document}

\section{Derivation}
HJB is
$$0 = \rho \theta V_t(\frac{C_t^{\frac{1-\gamma}{\theta}}}{((1-\gamma)V_t)^{\frac{1}{\theta}}}-1) + \frac{E[dV_t]}{dt}$$
Let's define $G_t$ such that
$$V_t = \frac{C_t^{1-\gamma}}{1-\gamma} G_t$$
By Ito, 
$$0 = \rho\theta(G_t^{1-\frac{1}{\theta}}-G_t)   +  G_t E\frac{\frac{dC_t^{1-\gamma}}{C_t^{1-\gamma}}}{dt} + E\frac{dG_t}{dt} + E\frac{dG_t\frac{dC_t^{1-\gamma}}{C_t^{1-\gamma}}}{dt}$$

Denoting $\mu_{C}$ and $\sigma_{C}$ the geometric drift of $C_t$, we have
\begin{align*}
	\frac{dC_{t}^{1-\gamma}}{C_{t}^{1-\gamma}}=((1-\gamma)\mu_{C}-\frac{1}{2}(1-\gamma)\gamma\sigma^{2}_{C})dt + (1-\gamma)\sigma_{C}dW_t
\end{align*}


Injecting this expression into HJB and denoting $\mu_G, \sigma_G$ the arithmetic drift and volatility of $C_t$
$$0 = \rho \theta (G_t^{1-\frac{1}{\theta}}-G_t)  + G_t ((1-\gamma) \mu_{C} - \frac{1}{2}(1-\gamma)\gamma\sigma_{C}'\sigma_{C}) +  \mu_G + \sigma_G'(1-\gamma)\sigma_{C}$$

\section{Long run risk model}
\subsection{Derivation}
We now assume that the evolution of consumption is driven by two state variables $\mu_{t}$ and $\sigma_{t}$:
\begin{align*}
	\frac{dC_{t}}{C_{t}} & =  \mu_{t}dt+\nu_{D}\sqrt{\sigma_{t}}dZ_{t}\\
	d\mu_{t} & =  \kappa_{\mu}(\bar{\mu}-\mu_{t})dt+\nu_{\mu}\sqrt{\sigma_{t}}dZ_{t}^{\mu}\\
	d\sigma_{t} & =  \kappa_{\sigma}(1-\sigma_{t})dt+\nu_{\sigma}\sqrt{\sigma_{t}}dZ_{t}^{\sigma}
\end{align*}
We write $G_t = G(\mu, \sigma)$ and we get the PDE
\begin{align*}
	0&= \rho \theta[G^{1-\frac{1}{\theta}}- G]+G((1-\gamma)\mu-\frac{1}{2}(1-\gamma)\gamma\nu_D^2\sigma)\\
	&+ \kappa_{\mu}(\bar{\mu}-\mu)\frac{\partial G}{\partial\mu}+  \kappa_{\sigma}(1-\sigma)\frac{\partial G}{\partial\sigma}\\
	&+\frac{1}{2}\nu_{\mu}^{2}\sigma\frac{\partial^{2}G}{\partial\mu^{2}}+\frac{1}{2}\nu_{\sigma}^{2}\sigma \frac{\partial^{2}G}{\partial\sigma^{2}}
\end{align*}
Boundary coundition  = reflectiving barrier
\begin{align*}
	\partial_\mu G(\underline{u}, \sigma) &= 0  \\
	\partial_\mu G(\overline{u}, \sigma) &= 0 \\
	\partial_\sigma G(u, \underline{\sigma}) &= 0  \\
	\partial_\sigma G(u, \overline{\sigma}) &= 0 
\end{align*}
We can solve this PDE through the following finite difference scheme:
\begin{align*}
	0&= \rho \theta[(G_{ij}^{n+1})^{1-\frac{1}{\theta}}- G_{ij}^{{n+1}}]+G_{ij}^{n+1}((1-\gamma)\mu_i-\frac{1}{2}(1-\gamma)\gamma\nu_D^2\sigma_j)\\
	&+\kappa_{\mu_i}(\bar{\mu}-\mu_i)^+(G_{i+1, j}^{n+1}-G_{i, j}^{n+1}) +\kappa_{\mu_i}(\bar{\mu}-\mu_i)^-(G_{i, j}^{n+1}-G_{i-1, j}^{n+1})\\
	&+\kappa_{\sigma_j}(1-\sigma_j)^+(G_{i, j+1}^{n+1}-G_{i,j}^{n+1}) +\kappa_{\sigma_j}(1-\sigma_j)^-(G_{i, j}^{n+1}-G_{i,j-1}^{n+1})\\
	&+\frac{1}{2}\nu_{\mu_i}^{2}\sigma_j(G_{i+1, j}^{n+1} - 2 G_{i, j}^{n+1} + G_{i-1, j}^{n+1})+\frac{1}{2}\nu_{\sigma_j}^{2}\sigma_j(G_{i, j+1}^{n+1} - 2 G_{i, j}^{n+1} + G_{i, j-1}^{n+1})
\end{align*}
with usual ghost nodes to satisfy boundary conditions.
The scheme satisfies the monotonicity condition of the Barles-Souganadis Theorem. 
\begin{itemize}
	\item monotonicity in $G_{i+1, j}_{i-1}^{n+1}, G_{i-1, j}_{i-1}^{n+1}, G_{i, j-1}_{i-1}^{n+1}, G_{i, j-1}_{i+1}^{n+1}$ by upwinding 
	\item monoticity in $G_{i, j}^{n}$  because \dots there is  no term in $G_{i, j}^{n}$. We do need a fully explicit scheme : if $(G_{ij}^{n+1})^{1-\frac{1}{\theta}}$ was replaced  by $(G_{ij}^{n})^{1-\frac{1}{\theta}}$, the scheme would not be decreasing in $G_{ij}^{n}$ for $\theta < 0$.
\end{itemize}
Contrary to the schemes of Achdou et al., 2014,  the scheme contains a non linear term in $G_{ij}^{n+1}$. We can solve this non-linear scheme by Newton method (first order taylor approximation of the non linear term), i.e. by iterating
\begin{align*}
	0&= \rho (G_{ij}^{n})^{1-\frac{1}{\theta}}+ \rho  (\theta-1) (G_{ij}^{n})^{-\frac{1}{\theta}}G_{ij}^{n+1}\\
	&- \rho \theta G_{ij}^{{n+1}}+G_{ij}^{n+1}((1-\gamma)\mu_i-\frac{1}{2}(1-\gamma)\gamma\nu_D^2\sigma_j)\\
	&+\kappa_{\mu_i}(\bar{\mu}-\mu_i)^+(G_{i+1, j}^{n+1}-G_{i, j}^{n+1}) +\kappa_{\mu_i}(\bar{\mu}-\mu_i)^-(G_{i, j}^{n+1}-G_{i-1, j}^{n+1})\\
	&+\kappa_{\sigma_j}(1-\sigma_j)^+(G_{i, j+1}^{n+1}-G_{i,j}^{n+1}) +\kappa_{\sigma_j}(1-\sigma_j)^-(G_{i, j}^{n+1}-G_{i,j-1}^{n+1})\\
	&+\frac{1}{2}\nu_{\mu_i}^{2}\sigma_j(G_{i+1, j}^{n+1} - 2 G_{i, j}^{n+1} + G_{i-1, j}^{n+1})+\frac{1}{2}\nu_{\sigma_j}^{2}\sigma_j(G_{i, j+1}^{n+1} - 2 G_{i, j}^{n+1} + G_{i, j-1}^{n+1})
\end{align*}
This defines a linear semi explicit scheme \footnote{Actually, the schemes used in Achdou et al., 2014 can be obtained in this way.}. Note that this scheme does not satisfy the monotonicity condition of the Barles-Souganadis Theorem since the derivative of the scheme wrt $G_{ij}^{n}$ is 
$$\rho (1- \theta)(G_{ij}^{n})^{-\frac{1}{\theta}} (1- \frac{G_{ij}^{n+1}}{G_{ij}^{n}})$$
Yet, this scheme converges because (i) the non linear fully implicit scheme satisfies Barles-Souganadis Theorem (ii) Newton method converges to the non linear scheme


Instead of the Newton method, we could use another off the shelf non linear solver. The NLsolve package in Julia uses the Powell Dog-leg method. In this case, the updating step is a mix of gradient and Newton steps. 

\subsection{Comparaison}

\begin{tabular}{|c|c|c|c|c|c|}
	\hline 
	Name & BY04 & BY04 & This paper & This paper & Link
	\\
	\hline 
	\hline 
	mean growth rate & $\mu$ & 0.0015 & $\bar{\mu}$ & 0.0015 & $\mu=\bar{\mu}$
	\\
	\hline 
	mean volatility & $\sigma^{2}$ & 0.00006084 & $\nu_{D}$ & 0.0078 & $\sqrt{\sigma^{2}}=\nu_{D}$
	\\
	\hline 
	growth persistence & $\rho$ & 0.979 & $\kappa_{\mu}$ & 0.0212 &  - \log(\rho) = \kappa_\mu 
	\\
	\hline 
	volatility persistence & $\nu_{1}$ & 0.987 & $\kappa_{\sigma}$ & 0.0131 & $-\log\left(\nu_{1}\right)=\kappa_{\sigma}$
	\\
	\hline 
	growth rate volatility & $\varphi_{e}$ & 0.044 & $\nu_{\mu}$ & 0.0003432 & $\varphi_{e}\times\sqrt{\sigma^{2}}=\nu_{\mu}$
	\\
	\hline 
	volatility volatility & $\sigma_{w}$ & 0.0000023 & $\nu_{\sigma}$ & 0.0378 & $\sigma_{w}/\sigma^{2}=\nu_{\sigma}$
	\\
	\hline 
	time discount & $\delta$ & 0.998 & $\rho$ & 0.002 & $-\log\left(\delta\right)=\rho$
	\\
	\hline 
	RRA & $1-\gamma$(RRA) & 7.5 or 10 & $1-\gamma$ & -6.5 or -9 & $1-\text{RRA}=1-\gamma$
	\\
	\hline 
	IES & $\psi$ & 1.5 & $\psi$ & 1.5 & \psi = \psi
	\\
	\hline
\end{tabular}
Also,  $\theta = (1-\gamma)/(1- 1/\psi)$ = -19.50 or -27.
Let's express the consumption to wealth ratio $k_t$ in term of state variables.
$$V = G_tk_t^{1-\gamma}\frac{W^{1-\gamma}}{(1-\gamma)}$$
FOC for consumption can be written
$$k_t = \rho^{\psi} k_t^{1-\psi}G_t^\frac{1-\psi}{1-\gamma}$$
General equilibrium gives
$$ k_t = \rho G_t^{-1/\theta}$$
Bansal Yaron find
\begin{align*}
	\frac{1}{\theta}\log G_t-\log \rho &\approx A_1 \mu_t + A_2 \nu_D^2\sigma_t\\
	A1 &= \frac{1-\frac{1}{\psi}}{1-0.997 e^{-\kappa_\mu}}\\
	A2 &= 0.5\theta\frac{(1 - \frac{1}{\psi})^2 + (A_1  \kappa_1  \frac{\nu_\mu}{\nu_D^2})^2}{1-0.997e^{-\kappa_\sigma}}
\end{align*}
\end{document}